\documentclass[usepdftitle=false]{beamer}

\usepackage{pgfpages}
% \setbeameroption{show notes on second screen}

\usetheme{Darmstadt}

\usecolortheme{beaver}

\usefonttheme[onlymath]{serif}

\usepackage{lmodern}
\usepackage[T1]{fontenc}
\usepackage[utf8]{inputenc}
\usepackage[ngerman]{babel}

\usepackage{hyperref}
\usepackage{caption}
\usepackage{csquotes}

\usepackage[backend=biber,style=numeric,sorting=none,url=false]{biblatex}
\addbibresource{bibliography.bib}

\usepackage{array}
\usepackage{mathtools}

\DeclareSymbolFont{letters}{OML}{ztmcm}{m}{it}

\usepackage{stmaryrd} % for contradiction lightning symbol

\usepackage{tikz}
\usetikzlibrary{automata,positioning,arrows,shapes,decorations,calc}

\pgfdeclarelayer{background}
\pgfdeclarelayer{foreground}
\pgfsetlayers{background,main,foreground}

% see https://tex.stackexchange.com/questions/20425/z-level-in-tikz/20426#20426
\makeatletter
\pgfkeys{%
  /tikz/on layer/.code={
    \pgfonlayer{#1}\begingroup
    \aftergroup\endpgfonlayer
    \aftergroup\endgroup
  },
  /tikz/node on layer/.code={
    \gdef\node@@on@layer{%
      \setbox\tikz@tempbox=\hbox\bgroup\pgfonlayer{#1}\unhbox\tikz@tempbox\endpgfonlayer\egroup}
    \aftergroup\node@on@layer
  },
  /tikz/end node on layer/.code={
    \endpgfonlayer\endgroup\endgroup
  }
}
\def\node@on@layer{\aftergroup\node@@on@layer}
\makeatother


% https://tex.stackexchange.com/questions/6135/how-to-make-beamer-overlays-with-tikz-node
\tikzset{onslide/.code args={<#1>#2}{%
		\only<#1>{\pgfkeysalso{#2}} % \pgfkeysalso doesn't change the path
}}


% https://tex.stackexchange.com/questions/66532/proofs-in-beamer
\makeatletter
\newenvironment<>{proofs}[1][\proofname]{%
    \par
    \def\insertproofname{#1\@addpunct{.}}%
    \usebeamertemplate{proof begin}#2}
  {\usebeamertemplate{proof end}}
\newenvironment<>{proofc}{%
  \setbeamertemplate{proof begin}{\begin{block}{}}
    \par
    \usebeamertemplate{proof begin}}
  {\usebeamertemplate{proof end}}
\newenvironment<>{proofe}{%
    \par
    \pushQED{\qed}
    \setbeamertemplate{proof begin}{\begin{block}{}}
    \usebeamertemplate{proof begin}}
  {\popQED\usebeamertemplate{proof end}}
\makeatother


\beamertemplatenavigationsymbolsempty
\newcounter{slideinframe}
\defbeamertemplate*{footline}{slide in frame number}{%
    \hfill%
    \usebeamercolor[fg]{page number in head/foot}%
    \usebeamerfont{page number in head/foot}%
    \ifnum\insertframeendpage>\insertframestartpage
        \setcounter{slideinframe}{\numexpr\thepage-\insertframestartpage+1}
        \insertframenumber{}\,-\,\theslideinframe
    \else
        \insertframenumber{}
    \fi%
    %\;/\;\inserttotalframenumber
        \kern1em\vskip2pt%
}
\setbeamertemplate{theorems}[numbered]


% \DeclareMathOperator{\Paths}{Paths}

\newcommand{\VC}{\textsc{Vertex Cover}}
\newcommand{\HC}{\textsc{Hamiltonian Cycle}}
\newcommand{\HP}{\textsc{Hamiltonian Path}}
\newcommand{\HPP}{\textsc{Hamiltonian Path Between Two Points}}
\newcommand{\LP}{\textsc{Longest Path}}


\title[Hamiltonian Cycle]{NP-Vollständigkeit von Hamiltonian Cycles}
\subtitle{durch Reduktion von Vertex Covern}
\author{Stefan Walter}
\institute{Universit\"a{}t Leipzig  \\
    \tiny 10-202-2112, Complexity Theory
}
\date{16. Dezember 2019}

\hypersetup{
    pdfinfo={
        Author={Stefan Walter},
        Title={NP-Completeness of Hamiltonian Cycles by reduction from Vertex Covers},
        Subject={Complexity Theory Seminar}
    }
}


\begin{document}

\frame[typeset second]{\only<second:0>{\titlepage}}

\begin{frame}{Literatur}
    \fullcite[Chapter 3.1.4, p.~56 to 60]{np}
    \note{größtenteils direkt übernommen}
\end{frame}


\section{Problemdefinitionen}

\begin{frame}
    \begin{exampleblock}{\VC{} (Knotenüberdeckung)}
        \alert{Eingabe:} Ein Graph $G = (V,E)$ und $K \in \mathbb{N}_+$ mit $K \leq |V|$.

        \alert{Ausgabe:} Existiert ein \alert{Vertex Cover} der Größe maximal $K$ für $G$?  \\
        D.\,h. eine Menge $V'\subseteq V$ mit
        \begin{itemize}
            \item $|V'| \leq K$
            \item $\displaystyle
                \forall \, \{u,v\} \in E: \; u \in V' \quad \lor \quad v \in V'
                $
        \end{itemize}
    \end{exampleblock}

    \begin{overprint}
        \onslide<2>\centering\includegraphics[width=5cm]{vc-example.pdf}
        \onslide<3->\centering\includegraphics[width=5cm]{vc-example_colored.pdf}
    \end{overprint}
    \note{
        \begin{itemize}
            \item \VC{} ist so eine Arte Erreichbarkeit aller Kanten durch eine bestimmte Menge von Knoten
            \item hier nur die Berechnungsvarianten
            \item hier ungerichtete Graphen; \\
                Problem für gerichtete auch in NP
        \end{itemize}
    }
\end{frame}

\begin{frame}
    \begin{exampleblock}{\HC{} (Hamiltonkreis)}
        \alert{Eingabe:} Ein Graph $G = (V,E)$

        \alert{Ausgabe:} Enthält $G$ einen \alert{Hamiltonian Cycle}?  \\
        D.\,h. eine Knoten-Folge $(v_1,v_2,\dots,v_n)$ mit
        \begin{itemize}
            \item $n = |V|$
            \item  $\forall \, i \in \mathbb{N}, 1\leq i < n: \; \{v_i,v_{i+1}\} \in E$
                und
                $\{v_n,v_1\} \in E$
        \end{itemize}
    \end{exampleblock}

    \begin{overprint}
        \onslide<2>\centering\includegraphics[width=3.3cm]{hc-example.pdf}
        \onslide<3->\centering\includegraphics[width=3.3cm]{hc-example_colored.pdf}
    \end{overprint}

    \note{
        \begin{itemize}
            \item Pfad, der alle Knoten enthält und im selben Knoten beginnt und endet
            \item ähnlich zum TSP, nur nicht gewichtet
            \item als Sprache formuliert: Menge aller Graphen mit HC
        \end{itemize}
    }
\end{frame}


\section{NP-Vollständigkeit von \HC}

\begin{frame}{Vorgehen beim Zeigen von NP-Vollständigkeit}
    \begin{exampleblock}{Vorgehen}
        \begin{itemize}
            \item $\HC \in NP$
            \item \HC{} ist NP-schwer:
                $\forall \, L' \in NP: \; L' \preceq_L \HC$
        \end{itemize}
    \end{exampleblock}

    Notizen:
    \begin{itemize}
        \item hier mit $\VC \preceq_L \HC$
        \item Annahme: \VC{} ist NP-vollständig
    \end{itemize}

    \note{
        \begin{itemize}
            \item alle Probleme in NP lassen sich darauf reduzieren

            \item VC ist np schwer, alle auf vc reduzieren,
                VC auf HC reduzieren => alle auf HZ reduzieren
        \end{itemize}
    }
\end{frame}


\begin{frame}{Beweisidee}
    \begin{proofs}[Beweisidee]
        $\HC{} \in NP$:
        \begin{itemize}
            \item rate Knotenfolge $(v_1,v_2,\dots,v_n) \in V^{|V|}$ nichtdeterministisch
            \item teste, ob Folge die Bedingung erfüllt \\
                \quad $\Rightarrow$ polynomiell
        \end{itemize}
    \end{proofs}
    \note{
        \begin{itemize}
            \item werden hier nicht strikt beweisen / erklären
            \item Ansatz / Konstruktion aus dem Buch zeigen
            \item dazu habe ich mal Beispiel entworfen, die werden wir durchgehen
        \end{itemize}
    }
\end{frame}

\begin{frame}{Reduktion}
    \begin{proofc}
        NP-schwer: $\VC \preceq_L \HC$

        \alert{Eingabe:} Graph $G = (V,E)$ und $K \in \mathbb{N}_+, K \leq |V|$

        \alert{Ausgabe:} Graph $G' = (V',E')$,
        sodass
        \[
            (G,K) \in \VC \iff G' \in \HC
        \]
    \end{proofc}
    \note{
        \begin{itemize}
            \item Umkodieren/Umformulieren/Transformieren, sodass neuer Graph genau dann HC hat,
                wenn alter Graph VC maximal K hat

            \item wenn wir HC lösen können, können wir VC lösen

            \item Logarithmischem Platz
        \end{itemize}
    }
\end{frame}

\colorlet{darkgreen}{green!60!black}
\tikzset{
    shorten >= 1pt,
    node distance = 1cm,
    place/.style={circle,thick,minimum size=2mm,draw},
    every path/.style={-,very thick},
    highlighted/.style={line width=4pt,darkgreen,on layer=foreground},
    route1/.style={},
    route2/.style={},
    route3/.style={},
}

\newcommand{\ename}{e}
\newcommand{\uname}{u}
\newcommand{\vname}{v}
\newcommand{\COVTEST}{
    \node[place] (\uname\ename1) {};
    \node[place] (\uname\ename2) [below=of \uname\ename1] {};
    \node[place] (\uname\ename3) [below=of \uname\ename2] {};
    \node[place] (\uname\ename4) [below=of \uname\ename3] {};
    \node[place] (\uname\ename5) [below=of \uname\ename4] {};
    \node[place] (\uname\ename6) [below=of \uname\ename5] {};

    \node[place] (\vname\ename1) [right=3cm of \uname\ename1] {};
    \node[place] (\vname\ename2) [below=of \vname\ename1] {};
    \node[place] (\vname\ename3) [below=of \vname\ename2] {};
    \node[place] (\vname\ename4) [below=of \vname\ename3] {};
    \node[place] (\vname\ename5) [below=of \vname\ename4] {};
    \node[place] (\vname\ename6) [below=of \vname\ename5] {};

    \node[left=0.2em of \uname\ename1] {(\uname,\ename,1)};
    \node[left=0.2em of \uname\ename2] {(\uname,\ename,2)};
    \node[left=0.2em of \uname\ename3] {(\uname,\ename,3)};
    \node[left=0.2em of \uname\ename4] {(\uname,\ename,4)};
    \node[left=0.2em of \uname\ename5] {(\uname,\ename,5)};
    \node[left=0.2em of \uname\ename6] {(\uname,\ename,6)};
    \node[right=0.2em of \vname\ename1] {(\vname,\ename,1)};
    \node[right=0.2em of \vname\ename2] {(\vname,\ename,2)};
    \node[right=0.2em of \vname\ename3] {(\vname,\ename,3)};
    \node[right=0.2em of \vname\ename4] {(\vname,\ename,4)};
    \node[right=0.2em of \vname\ename5] {(\vname,\ename,5)};
    \node[right=0.2em of \vname\ename6] {(\vname,\ename,6)};

    \draw[on layer=background]
        (\uname\ename1) edge [route1,route2,route3]    (\uname\ename2)
        (\uname\ename2) edge [route1,route2,route3]    (\uname\ename3)
        (\uname\ename3) edge [route2,route3]           (\uname\ename4)
        (\uname\ename4) edge [route1,route2,route3]    (\uname\ename5)
        (\uname\ename5) edge [route1,route2,route3]    (\uname\ename6)

        (\vname\ename1) edge [route1,route2,route3]    (\vname\ename2)
        (\vname\ename2) edge [route1,route2,route3]    (\vname\ename3)
        (\vname\ename3) edge [route1,route2]           (\vname\ename4)
        (\vname\ename4) edge [route1,route2,route3]    (\vname\ename5)
        (\vname\ename5) edge [route1,route2,route3]    (\vname\ename6)

        (\uname\ename3) edge [route1]                  (\vname\ename1)
        (\vname\ename3) edge [route3]                  (\uname\ename1)
        (\uname\ename6) edge [route3]                  (\vname\ename4)
        (\vname\ename6) edge [route1]                  (\uname\ename4)
    ;
}
\newcommand{\COVTESTINOUT}{
    \draw
        (\uname\ename1) edge[bend right,dashed,route1,route2]  ++(-4em,4em)
        (\uname\ename6) edge[bend left,dashed,route1,route2]   ++(-4em,-4em)
        (\vname\ename1) edge[bend left,dashed,route2,route3]   ++(4em,4em)
        (\vname\ename6) edge[bend right,dashed,route2,route3]  ++(4em,-4em)
    ;
}


\begin{frame}{Konstruktions-Idee}
    \begin{proofc}
        \begin{minipage}{0.65\textwidth}
            \begin{itemize}
                \item $K$ "`Selektor"'-Knoten $a_1, a_2, \dots, a_K$
                \item $|E|$ "`Cover-Testing"'-Komponenten (abgekürzt CTK)
            \end{itemize}

            \pause

            \vspace*{1cm}
            z.\,B. für $e = \{u,v\} \in E$
        \end{minipage}\hfill\begin{minipage}{0.30\textwidth}
            \begin{center}
                \scalebox{0.5}{
                    \begin{tikzpicture}
                        \COVTEST
                        \COVTESTINOUT
                    \end{tikzpicture}
                }
            \end{center}
        \end{minipage}
    \end{proofc}

    \note{
        \begin{itemize}
            \item für jede Kante eine Cover-Testing-Komponente
        \end{itemize}
    }
\end{frame}


\begin{frame}{Konstruktions-Idee}
    \begin{proofc}
        Pfade durch Cover-Testing-Komponente bei Hamiltonkreis in $G'$: \\[1em]

        \scalebox{0.44}{
            \begin{tikzpicture}[
                    route1/.style={highlighted}
                ]
                \COVTEST
                \COVTESTINOUT
            \end{tikzpicture}
        }
        \hfill
        \scalebox{0.44}{
            \begin{tikzpicture}[
                    route2/.style={highlighted}
                ]
                \COVTEST
                \COVTESTINOUT
            \end{tikzpicture}
        }
        \hfill
        \scalebox{0.44}{
            \begin{tikzpicture}[
                    route3/.style={highlighted}
                ]
                \COVTEST
                \COVTESTINOUT
            \end{tikzpicture}
        }
    \end{proofc}

    \note{
        \begin{itemize}
            \item so aufgebaut, dass jeder HC in $G'$ die nur auf bestimmte
                Weise durchlaufen kann
            \item CTK werden über 4 Knoten mit anderen CTK und Selektor verbunden
            \item HC geht durch alle Knoten, also auch durch alle in jeder CTK
            \item z.B. nicht ue1 -> ve6 raus
            \item Möglichkeiten entsprechen Knoten im Vertex Cover:
                \begin{itemize}
                    \item u und nicht v
                    \item u und v
                    \item v und nicht u
                \end{itemize}
        \end{itemize}
    }
\end{frame}


\begin{frame}{Vollständige Reduktions-Konstruktion}
    \begin{proofc}
        $G' = (V',E')$ mit
        \[
            \begin{aligned}[t]
                V' &=  \underbrace{
                        \left\{ a_i \mid 1 \leq i \leq K \right\} 
                    }_{\text{Selektor}}
                    \cup \underbrace{
                        \left( \bigcup_{e\in E} V'_e \right)
                    }_{\text{Cover-Testing}}  \\
                E' &= \underbrace{
                        \left( \bigcup_{e \in E} E'_e \right)
                    }_{\text{Cover-Testing}}
                    \cup \left( \bigcup_{v \in V} E'_v \right) \cup E''
            \end{aligned}
        \]
    \end{proofc}
    \note{
        \begin{itemize}
            \item erstmal viel, aber explizit auf einmal
            \item Selektoren
        \end{itemize}
    }
\end{frame}


\begin{frame}{Vollständige Reduktions-Konstruktion}
    \begin{proofc}
        (für $e = \{ u,v \}  \in E$)
        \[
            \begin{aligned}[t]
                V'_e &= \left\{ (u,e,i), (v,e,i) \mid 1 \leq i \leq 6 \right\}  \\
                E'_e &= \Big\{ \{ (u,e,i), (u,e,i+1) \}, \{ (v,e,i), (v,e,i+1) \} \mid 1 \leq i \leq 5 \Big\}  \\
                     &\qquad \cup \Big\{ \{ (u,e,3), (v,e,1) \}, \{ (v,e,3), (u,e,1) \} \Big\}  \\
                     &\qquad \cup \Big\{ \{ (u,e,6), (v,e,4) \}, \{ (v,e,6), (u,e,4) \} \Big\}  \\
                E'_v &= \Big\{ \{ (v,e_{v[i]},6), (v,e_{v[i+1]},1) \} \mid 1 \leq i < deg(v) \Big\}  \\
                E'' &= \Big\{ \{ a_i, (v,e_{v[1]},1) \}, \; \{ a_i, (v, e_{v[deg(v)]},6) \} \mid 1 \leq i \leq K, v \in V \Big\}
            \end{aligned}
        \]
    \end{proofc}
    \note{
        \begin{itemize}
            \item $V'_e$ und $E'_e$ CTK $\implies$ entspricht Kanten
            \item $E'_v$ und $E''$ Indizes: für jeden Knoten alle benachbarten Kanten irgendwie durchnummerieren / ordnen
            \item $E'_v$ Verbindungen CTK untereinander $\implies$ entspricht Knoten / Struktur  \\
                wir verbinden alle CTKs die im ursprünglichen $G$ benachbarte
                Kanten sind, also vom selben Knoten ausgehen
            \item $E''$ Verbindungen CTK und Selektor  \\
                durch $E'_v$ teilverbundene CTKs, die Enden dieser verbinden wir mit Selektoren  \\
                mit jedem Selektor
        \end{itemize}
    }
\end{frame}


\begin{frame}{Konstruktion Beispiel $\in VC$}
    \begin{exampleblock}{}
        \begin{minipage}{0.15\textwidth}
            \includegraphics[width=\linewidth]{construction_in_vc.pdf}

            \begin{overprint}
                \onslide<3-4>$V'_e, E'_e$
                \onslide<5>$E'_v$
                \onslide<6>$a_i$
                \onslide<7>$E''$
                \onslide<8>$G' \in HC$
            \end{overprint}
        \end{minipage}\hfill\begin{minipage}{0.85\textwidth}
            \begin{overprint}
                \onslide<1>\[
                        G = (V,E) = \Big( \{a,b,c\}, \Big\{ \{a,b\}, \{b,c\} \Big\} \Big) \qquad K=1
                    \]
                \onslide<2->\[
                        G = (V,E) = \Big( \{a,b,c\}, \Big\{ \underbrace{\{a,b\}}_{e_1}, \underbrace{\{b,c\}}_{e_2} \Big\} \Big) \qquad K=1
                    \]
            \end{overprint}

            \begin{center}
                \scalebox{0.45}{
                    \begin{tikzpicture}[
                            COV/.pic={\COVTEST},
                            COVROT/.pic={
                                \begin{scope}[rotate=90,transform shape]
                                    \COVTEST
                                \end{scope}
                            },
                            every path/.append style={opacity=0},
                            rounded corners=10pt
                        ]
                        % place node at top left corner with a small margin to prevent the image from shifting on highlighting otherwise outer paths
                        % ugly hack, but I don't know better ...
                        \node at (-2.8,1) {};

                        \begin{scope}[
                                every path/.append style={onslide=<3->{opacity=1}},
                                route3/.style={onslide=<8->{highlighted}}
                            ]
                            \renewcommand{\ename}{e1}\renewcommand{\uname}{a}\renewcommand{\vname}{b}
                            \pic at (0,0) {COV};
                        \end{scope}

                        \begin{scope}[
                                every path/.append style={onslide=<4->{opacity=1}},
                                route1/.style={onslide=<8->{highlighted}}
                            ]
                            \renewcommand{\ename}{e2}\renewcommand{\uname}{b}\renewcommand{\vname}{c}
                            \pic at (10,0) {COV};
                        \end{scope}

                        \node[place,below=3cm,onslide=<6->{opacity=1}] (a1) at ($(be16)!0.5!(be26)$) {$a_{1}$};

                        % E'_v
                        \begin{scope}[
                                every path/.append style={onslide=<5->{opacity=1}},
                                every node/.style={}
                            ]
                            \draw[onslide=<8->{highlighted}]
                                (be16) -- ++(0,-0.5)    -- ++(2,0) coordinate (temp1)
                                (be21) -- ++(0,0.5)     -- ++(-2,0) coordinate (temp2)
                                (temp1) -- (temp2)
                            ;
                        \end{scope}

                        % E''
                        \begin{scope}[
                                every path/.append style={onslide=<7->{opacity=1}}
                            ]
                            \draw
                                (ae11) -- ++(0,0.5)     -- ++(-2,0)     -- ++(0,-8.5)   -- (a1)
                                (ae16) -- ++(0,-0.5)    -- (a1)
                                (ce21) -- ++(0,0.5)     -- ++(3,0)      -- ++(0,-10)    -- (a1)
                                (ce26) -- ++(0,-0.5)    -- (a1)
                            ;
                            \draw[onslide=<8->{highlighted}]
                                (be11) -- ++(0,1)       -- ++(-6,0)     -- ++(0,-10)    -- (a1)
                                (be26) -- ++(0,-0.5)    -- (a1)
                            ;
                        \end{scope}
                    \end{tikzpicture}
                }
            \end{center}
        \end{minipage}
    \end{exampleblock}

    \note{
        \begin{itemize}
            \item Kanten nummerieren
            \item CTK für e1 und e2
            \item $E'_v$: verbinden CTKs aller benachbarten Kanten, \\
                hier nur bei b
            \item Selektor-Knoten
            \item $E''$ verbinden aller noch offenen Knoten in CTK mit jedem Selektor-Knoten
            \item Ablesen von VC über Pfad:  \\
                später sehen wir: jeder Pfad in HK von Selektor zurück entspricht einem Knoten in $G$  \\
                also hier nur eine Möglichkeit: b
        \end{itemize}
    }
\end{frame}


\begin{frame}{Konstruktion Beispiel $\not\in VC$}
    \begin{exampleblock}{}
        \begin{minipage}{0.13\textwidth}
            \includegraphics[width=\linewidth]{construction_not_in_vc.pdf}

            \begin{overprint}
                \onslide<3-5>$V'_e, E'_e$
                \onslide<6>$E'_v$
                \onslide<7>$a_i$
                \onslide<8>$E''$
                \onslide<9->$G' \not\in HC$
            \end{overprint}
        \end{minipage}\hfill\begin{minipage}{0.87\textwidth}
            \begin{overprint}
                \onslide<1>\[
                        G = (V,E) = \Big( \{a,b,c\}, \Big\{ \{a,b\}, \{b,c\}, \{c,a\} \Big\} \Big) \quad K=1
                    \]
                \onslide<2->\[
                        G = (V,E) = \Big( \{a,b,c\}, \Big\{
                            \underbrace{\{a,b\}}_{e_1},
                            \underbrace{\{b,c\}}_{e_2},
                            \underbrace{\{c,a\}}_{e_3}
                        \Big\} \Big) \quad K=1
                    \]
            \end{overprint}

            \begin{center}
                \scalebox{0.32}{
                    \begin{tikzpicture}[
                            COV/.pic={\COVTEST},
                            every path/.append style={opacity=0},
                            rounded corners=10pt
                        ]
                        \begin{scope}[
                                rotate=90,transform shape,
                                % xscale=-1,
                                every path/.append style={onslide=<3->{opacity=1}},
                            ]
                            \renewcommand{\ename}{e1}\renewcommand{\uname}{a}\renewcommand{\vname}{b}
                            \pic at (0,0) {COV};
                        \end{scope}

                        \begin{scope}[
                                rotate=90,transform shape,
                                every path/.append style={onslide=<4->{opacity=1}},
                            ]
                            \renewcommand{\ename}{e2}\renewcommand{\uname}{b}\renewcommand{\vname}{c}
                            \pic at (0,-10) {COV};
                        \end{scope}

                        \begin{scope}[
                                rotate=90,transform shape,
                                every path/.append style={onslide=<5->{opacity=1}},
                            ]
                            \renewcommand{\ename}{e3}\renewcommand{\uname}{c}\renewcommand{\vname}{a}
                            \pic at (0,-20) {COV};
                        \end{scope}

                        \node[place,below=5cm,onslide=<7->{opacity=1}] (a1) at ($(be23)!0.5!(be24)$) {$a_{1}$};

                        % E'_v
                        \begin{scope}[
                                every path/.append style={onslide=<6->{opacity=1}}
                            ]
                            \draw
                                (ae16)  -- ++(0.5,0)   -- ++(0,-2.5)    -- ++(10.5,0)   -- (ae31)
                                (be16)  -- (be21)
                                (ce26)  -- (ce31)
                            ;
                        \end{scope}

                        % E''
                        \begin{scope}[
                                every path/.append style={onslide=<8->{opacity=1}}
                            ]

                            \draw
                                (ae11)  -- ++(-0.5,0)   -- ++(0,-2)     -- (a1)
                                (ae36)  -- ++(1,0)      -- ++(0,-6)     -- (a1)
                                (be11)  -- ++(-1,0)     -- ++(0,-6)     -- (a1)
                                (be26)  -- ++(0.5,0)    -- ++(0,-2)     -- (a1)
                                (ce21)  -- ++(-0.5,0)   -- ++(0,2)      -- ++(-11,0)    -- ++(0,-9)     -- (a1)
                                (ce36)  -- ++(0.5,0)    -- ++(0,-2)     -- (a1)
                            ;
                        \end{scope}
                    \end{tikzpicture}
                }
            \end{center}
        \end{minipage}
    \end{exampleblock}
    \note{
        \begin{itemize}
            \item kein HC, da: beginnend in a1 alle Kanten durchtesten:  \\
                immer eine CTK nicht drin
        \end{itemize}
    }
\end{frame}


\begin{frame}{Erklärungsversuch Korrektheit Reduktion}
    \begin{proofc}
        \begin{itemize}
            \item Konstruktion von $G'$ aus $G$ und $K$ in logarithmischem Platz
                \note{erklären! -- Ausgabeband unbeschränkt, über alle Knoten / Kanten iterieren}
            \item Klarmachen, dass Reduktion korrekt ist
        \end{itemize}
    \end{proofc}
\end{frame}


\begin{frame}{Erklärungsversuch Korrektheit Reduktion}
    \note{
        \begin{itemize}
            \item nicht strikt, sehr informell
            \item bisschen abgewandelt vom Buch
        \end{itemize}
    }
    \begin{proofc}
        \[
            (G,K) \in \VC \impliedby G' \in \HC
        \]
        \pause

        Existiere ein Hamiltonkreis in $G'$.  \pause
        CTKs in $G'$ entsprechen Kanten in $G$ (bijektiv).  \pause
        Betrachte alle Teilpfade, die in einem Selektor-Knoten in $G'$ beginnen und enden und zwischendurch keinen solchen beinhalten.  \pause
        Jeder solche Pfad entspricht bijektiv einem Knoten in $G$; und alle darauf liegenden CTKs entsprechen genau den an diesem Knoten anliegenden Kanten in $G$.  \pause
        Da im Hamiltonkreis alle Knoten in $G'$ enthalten sind (damit auch alle CTKs), sind in $G$ alle den CTKs entsprechenden Kanten durch die entsprechenden Knoten erreichbar.  \pause
        Also existiert ein Vertex-Cover der Größe maximal $K$ für $G$.
    \end{proofc}
\end{frame}


\begin{frame}{Erklärungsversuch Korrektheit Reduktion}
    \begin{proofc}
        \[
            (G,K) \in \VC \implies G' \in \HC
        \]
        \pause

        Sei $V^* \subseteq V$ ein Vertex-Cover für $G$ mit $|V^*| \leq K$.  \pause
        Sei o.B.d.A. $|V^*| = K$ mit $V^* = \{v_1,v_2,\dots,v_K\}$  (im Zweifel minimales Cover mit weiteren Knoten auffüllen).
    \end{proofc}
\end{frame}


\begin{frame}{Erklärungsversuch Korrektheit Reduktion}
    \begin{proofc}
        Die folgenden Kanten bilden einen Hamiltonkreis:  \pause
        \begin{itemize}[<+->]
            \item $\forall \, e = \{u,v\} \in E$ die entsprechenden Kanten innerhalb CTKs; 
                je nachdem, ob $\{u,v\} \cap V^*$ gleich $\{u\}$, $\{u,v\}$ oder $\{v\}$ ist  \\
                (ein Fall trifft immer zu, da $V^*$ Vertex-Cover ist)

            \item Verbindungen zwischen den CTKs:
                $\displaystyle
                    \forall \, 1 \leq i \leq K: \; E_{v_i}'
                $

            \item Verbindungen zw. Selektor-Knoten und Beginn CTK-Ketten:
                \[
                    \forall \, 1 \leq i \leq K: \; \{ a_i, (v_i, e_{v_i [1]}, 1) \}
                \]

            \item Verbindungen zw. Selektor-Knoten und Ende CTK-Ketten:
                \[
                    \begin{aligned}[t]
                        \forall \, 1 \leq i < K: \; \{ a_{i+1}, (v_i, e_{v_i [deg(v_i)]}, 6) \}  \\
                        \{ a_1, (v_K, e_{v_K [deg(v_K)]}, 6) \}
                    \end{aligned}
                \]
        \end{itemize}
    \end{proofc}

    \note{
        \begin{itemize}
            \item explizite angabe
            \item so teilverbundene CTK Ketten
            \item für jeden Knoten in VC

            \item ohne Beweis, klarmachen
            \item Erläutern, ist genau der HC im Beispiel
        \end{itemize}
    }
\end{frame}


\section{Ausblick}

\begin{frame}{Varianten}
    \begin{itemize}
        \item \HP{}: \\
            wie \HC{}, nur ohne Bedingung $\{v_n,v_1\} \in E$

        \item \HPP{}: \\
            wie \HP{}, nur expliziter Start- und Endknoten
    \end{itemize}

    $\Rightarrow$ alle NP-vollständig

    \note{
        \begin{itemize}
            \item kann mit leichter Änderung der Konstruktion gezeigt werden
        \end{itemize}
    }
\end{frame}


\begin{frame}{\LP}
    \begin{exampleblock}{\LP{} -- Berechnungsvariante}
        \alert{Eingabe:} Ein Graph $G = (V,E)$ und $K \in \mathbb{N}_+, K \leq |V|$.

        \alert{Ausgabe:} Enthält $G$ einen Pfad ohne Knotenwiederholung mit mindestens $K$ Kanten?
    \end{exampleblock}

    Beweisidee: Reduktion von \HP{} mit:
    \begin{gather*}
    (V,E) \in \HP  \\
    \iff  \\
    ((V,E),|V|-1) \in \LP
    \end{gather*}

    \vfill
    \pause

    \begin{exampleblock}{\LP{} -- Optimierungsvariante}
        \alert{Eingabe:} Ein Graph $G = (V,E)$.

        \alert{Ausgabe:} Längster Pfad ohne Knotenwiederholung in $G$.
    \end{exampleblock}

    \note{
        \begin{itemize}
            \item ähnliches Problem
            \item unintuitiv, da Shortest Path zw. zwei Knoten in P (Dijkstra, $A^*$)
            \item wirkt intuitiv sei, als seie das nützlichere Problem leichter zu lösen,  \\
                schön, aber unerwartet
        \end{itemize}
    }
\end{frame}

\begin{frame}{Zusammenfassung}
    \begin{itemize}
        \setlength{\itemsep}{1em}
        \item \HC{} ist NP-vollständig
        \item Reduktion von \VC{}
        \item Varianten und verwandte Probleme
    \end{itemize}
\end{frame}

\end{document}
